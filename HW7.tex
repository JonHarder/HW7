\documentclass[11pt, letterpaper]{article}
\usepackage[margin=1in]{geometry}

\author{Jon Harder \& Zach Dunn}
\date{\today}
\title{Homework 7}
\begin{document}
\maketitle
\section{Undecidability}

\begin{verbatim}
# HALT(M,w) -> P(something) -> ALWAYSYES(P)
P(x):
   if x == 0:
      M(w); return "yes"
   else return "yes"
\end{verbatim}


\section{NP-Completeness}
To reduce $SAT$ to $STINGYSAT$, we can take $\phi$ and pass it unchanged to $STINGYSAT$.  To
generate $K$, we set it to the number of unique variables present in $\phi$. The $SAT$ problem
therefore is a more specific instance of $STINGYSAT$ in that $STINGYSAT$ solves a number of problems
where $SAT$ is only one particular instance.  Because $STINGYSAT$ is a more general problem which
does more work, we can say $STINGYSAT$ is at least as hard as $SAT$ and thus $SAT \leq STINGYSAT$
and because $SAT$ is NP-complete, $STINGYSAT$ must be as well.\\

\section{Decision vs. Search}

\end{document}
