\documentclass[11pt, letterpaper]{article}

\usepackage[margin=1in]{geometry}

\newcommand{\solution}[1]{\noindent{\textbf{Solution #1:}}}
\newcommand{\myskip}{\vskip 0.5in}

\title{CISC 490---Homework \#7 Solutions}
\author{Jon Harder \& Zach Dunn}
\date{\today}

\begin{document}
\maketitle

\solution{1} \textbf{Undecidability.}

The \texttt{HALT} program is ``simply'' tasked with determining whether a given
program halts (i.e., eventually stops executing) on a given input. To help us
solve this problem, we are given the program \texttt{ALWAYSYES} which takes a
program and tells us whether that program outpus ``yes'' for every input. If we
want to know whether \texttt{M} halts on \texttt{w}, we have to manipulate it
somehow so that the \texttt{ALWAYSYES} program can answer our question. To know
whether \texttt{M} halts on \texttt{w} we simply have to run it once, but how
can we fit this into a program that will comply with \texttt{ALWAYSYES}'s rules?
We simply output ``yes'' for every input except for one. In that one instance
where we do not simply return, we run \texttt{M} on \texttt{w}, then after that
we return ``yes.'' If indeed \texttt{M} halts on \texttt{w}, then
\texttt{ALWAYSYES} will itself return ``yes'' because all inputs---including the
one we picked out---returned ``yes.'' If, however, \texttt{M} does not halt on
\texttt{w}, then our program \texttt{P} will never reach the point where it
returns ``yes'' and so \texttt{ALWAYSYES} will return ``no,'' thus solving the
halting problem.

This is a problem though, because we know from class that solving the halting
problem is impossible; therefore, any program that solves the halting problem
must itself be impossible. Below is the function we chose to be put into the
\texttt{ALWAYSYES} program:

\begin{verbatim}
P(x):
    if x == 0:
        M(w)
        output "yes"
    else:
        output "yes"
\end{verbatim}

\myskip

\solution{2} \textbf{NP-Completeness.}

To reduce $SAT$ to $STINGYSAT$, we can take $\phi$ and pass it unchanged to
$STINGYSAT$. To generate $K$, we set it to the number of unique variables
present in $\phi$. The $SAT$ problem therefore is a more specific instance of
$STINGYSAT$ in that $STINGYSAT$ solves a number of problems where $SAT$ is only
one particular instance.  Because $STINGYSAT$ is a more general problem which
does more work, we can say $STINGYSAT$ is at least as hard as $SAT$ and thus
$SAT \leq STINGYSAT$ and because $SAT$ is NP-complete, $STINGYSAT$ must be as
well.

\myskip

\solution{3} \textbf{Decision vs. Search.}

The first task in the algorithm is to find the unique variables in the given
clause $\phi$. To do this we use the python \texttt{set} data structure, which
holds unique values. At this point, we do not care about whether the variable
is negated or not so we use the built-in \texttt{abs} function to prevent adding
$1$ and $-1$ to our set of variables, since those are actually the same, only
it is negated. Below is our function for finding the unique variables:

\begin{verbatim}
def findvars(formula):
    thevars = set([])
    for l in formula:
        for item in l:
            thevars.add(abs(item))

    return list(thevars)
\end{verbatim}

The second utility function is called \texttt{setvartrue}. This function will
manipulate the formula to make a certain expression in a clause be true. It is
slightly different from making the variable itself be true, in that forcing $-1$
to be true means that $1$ will actually be false so that ``not $1$'' will
evaluate to true. Over the course of setting variables to true, we may create a
clause that will always be false, and therefore make $\phi$ unsatisfiable. In
our formula this is represented by an empty clause, because once we force an
expression to be true, we eliminate its opposite from all the clauses. As an
example, if we force $3$ to be true in the formula, then all instances of $-3$
must be false since ``not true'' is obviously false. Since these opposite
expressions effectively do nothing for the clause, we remove them. Removing them
is beneficial since if all of the expressions are false we will have an empty
list in our formula which is simple to check for. As we said before, if we end
up with an empty clause we know that those values for the variables do not
satisfy the original formula.

Here is our implementation:

\begin{verbatim} 
def setvartrue(formula, x):
    f = copy.deepcopy(formula)
    for l in f:
        for item in l:
            if item == -x:
                l.remove(item)

        if len(l) == 0:
            return `UNSAT'

    return f
\end{verbatim}

The final function is a basic way to implement the actual solving functionality
of a \texttt{SAT}-solver. The algorithm loops through the unique variables,
testing them against the formula by forcing them to be true or false by using
our \texttt{setvartrue} function from earlier. This function returns a whole new
formula that assumes the truth of the current variable. We then use
\textit{that} formula as the basis for testing the next variable. Once we decide
what a certain variable's value should be (because \texttt{setvartrue} did not
return \texttt{`UNSAT'}), we add that variable to the \texttt{ans} list which we
return at the end as the proper values which will satisfy the given formula.

Ideally we should never return \texttt{`UNSAT'} from within the \texttt{for}
loop because if we make it to the loop, that means the \texttt{SAT}-solver
library has determined the formula to be satisfiable. Here is our
implementation:

\begin{verbatim}
def findsatisfying(formula):
    if not satdecide(formula):
        return `UNSAT'

    ans = []
    variables = findvars(formula)
    form = copy.deepcopy(formula)
    for var in variables:
        f = setvartrue(form, var)
        if f == `UNSAT':
            f = setvartrue(form, -var)
            if f == `UNSAT':
                return `UNSAT'
            else:
                ans.append(-var)
                form = f
        else:
            ans.append(var)
            form = f

    return ans
\end{verbatim}
\end{document}
